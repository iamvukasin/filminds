\section{Језици и радни оквири}

\subsection{\textit{Front-end}}

Изглед сервиса је базиран на радном оквиру \textit{Material Web Components},
уз прилагођавања и стилске модификације како би се прилагодио захтевима и
потребама пројекта.

\textit{Build system} који је коришћен је \textit{Gulp} 4.0.0 , који је задужен
за превођење \textit{Sass} стилова, минификацију и спајање \textit{JavaScript} и
CSS скрипти, као и за компресију слика и покретање локалног сервера са опцијом
аутоматског учитавања станица приликом сваке промене HTML, CSS или
\textit{JavaScript} фајлова.

\textit{Yarn} 1.16.0 је коришћен као \textit{dependency manager}, како за
библиотеке коришћене на страницама, тако и за библиотеке коришћене за развој
\textit{front-end} дела пројекта.

Списак свих библиотека (ажурна верзија је у \verb|packages.json| у кореном
фолдеру пројекта):

\vspace{0.2cm}
\setlist{nolistsep}
\begin{itemize}[noitemsep]
    \item \verb|@material/button| (2.0.0) -- \textit{Material Web Components},
    \item \verb|@material/card| (2.0.0) -- \textit{Material Web Components},
    \item \verb|@material/dialog| (2.0.0) -- \textit{Material Web Components},
    \item \verb|@material/drawer| (2.0.0) -- \textit{Material Web Components},
    \item \verb|@material/elevation| (1.1.0) --
        \textit{Material Web Components},
    \item \verb|@material/icon-button| (2.0.0) --
        \textit{Material Web Components},
    \item \verb|@material/layout-grid| (0.41.0) --
        \textit{Material Web Components},
    \item \verb|@material/select| (2.2.0) -- \textit{Material Web Components},
    \item \verb|@material/textfield| (2.1.0) --
        \textit{Material Web Components},
    \item \verb|@material/theme| (1.1.0) -- \textit{Material Web Components},
    \item \verb|@material/toolbar| (2.0.0) -- \textit{Material Web Components},
    \item \verb|@material/top-app-bar| (2.1.0) --
        \textit{Material Web Components},
    \item \verb|@material/typography| (1.0.0) --
        \textit{Material Web Components},
    \item \verb|chartist| (0.11.0) -- графикони,
    \item \verb|js-cookie| (2.2.0) -- рад са колачићима
\end{itemize}

\vspace{0.4cm}

Списак коришћених библиотека за развој (углавном додаци за \textit{Gulp} и
\textit{Babel}; ажурна верзија је у \verb|packages.json| у кореном фолдеру
пројекта):

\vspace{0.2cm}
\setlist{nolistsep}
\begin{itemize}[noitemsep]
    \item \verb|@babel/core| (7.3.4) -- модерни \textit{JavaScript} преводилац,
    \item \verb|@babel/plugin-proposal-optional-chaining| (7.2.0) --
        подршка за \textit{JavaScript optional chaining},
    \item \verb|@babel/preset-env| (7.3.4) -- подразумено \textit{Babel}
        окружење,
    \item \verb|babelify| (10.0.0) -- \textit{browserify} за \textit{Babel},
    \item \verb|browser-sync| (2.26.3) -- поново учитавање станица после
        промене,
    \item \verb|browserify| (16.2.3) -- \textit{Node.JS} стил писања,
    \item \verb|del| (4.0.0) -- брисање фолдера,
    \item \verb|gulp| (4.0.0) -- \textit{build system},
    \item \verb|gulp-autoprefixer| (6.1.0) -- \textit{Gulp} додатак за
        \textit{Autoprefixer},
    \item \verb|gulp-babel| (8.0.0-beta.2) -- \textit{Gulp} додатак за
        \textit{Babel},
    \item \verb|gulp-csso| (3.0.1) -- \textit{Gulp} додатак за оптимизацију
        CSS-а,
    \item \verb|gulp-sass| (4.0.2) -- \textit{Gulp} додатак за \textit{Sass},
    \item \verb|gulp-svgo| (2.1.0) -- \textit{Gulp} додатак за оптимизацију
        SVG-а,
    \item \verb|gulp-uglify| (3.0.2) -- \textit{Gulp} додатак за минификацију,
    \item \verb|vinyl-buffer| (1.0.1) -- спајање више скрипти,
    \item \verb|vinyl-source-stream| (2.0.0) -- спајање више скрипти
\end{itemize}

\subsection{\textit{Back-end}}

Као језик за \textit{back-end} коришћен је \textit{Python} 3.6, док је као радни
оквир коришћен \textit{Django} 2.2.1. \textit{Pipenv} је коришћен као
\textit{dependency manager} за \textit{Python} библиотеке.

Списак свих библиотека (ажурна верзија је у \verb|Pipenv.lock| у кореном
фолдеру пројекта):

\vspace{0.2cm}
\setlist{nolistsep}
\begin{itemize}[noitemsep]
    \item \verb|django| (2.2.1) -- \textit{back-end} радни оквир,
    \item \verb|psycopg2-binary| (2.8.2) -- подршка за рад са \textit{Postgres}
        базом,
    \item \verb|tmdbsimple| (2.2.0) -- омотач за \textit{The Movie Database}
        API,
    \item \verb|django-widget-tweaks| (1.4.3) -- додатни \textit{Django} тагови,
    \item \verb|djangorestframework| (3.9.4) -- \textit{Django} подршка за REST,
    \item \verb|wit| (5.1.0) -- препознавање писаног језика
\end{itemize}

\vspace{0.4cm}

Списак коришћених библиотека за развој (ажурна верзија је у \verb|Pipenv.lock| у
кореном фолдеру пројекта):

\vspace{0.2cm}
\setlist{nolistsep}
\begin{itemize}[noitemsep]
    \item \verb|pytest| (4.5.0) -- радни оквир за тестирање,
    \item \verb|pytest-flake8| (1.0.4) -- додатак за \textit{Pytest} за проверу
        стила кода (PEP8)
\end{itemize}

\subsection{База података}

\textit{Postgres} 11.3 је база података коришћена за чување корисничких
података.
