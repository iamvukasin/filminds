\section{Увод}

\subsection{Намена}

База података за пројекат \textit{Filminds} представља флексибилан и поуздан начин чувања 
података и приступа истим од стране веб сервера ради генерисања веб страница.
У документу су дати IE модел података, шема релационе базе података, као и опис
табела у бази података. Овај документ служи као основа за развој детаљне пројектне 
спецификацијепосматраног подсистема, имплементацију и тестирање. 

\subsection{Циљне групе}

Документ је намењен вођи пројекта и члановима развојног тима. Вођи пројекта
документ служи за планирање развојних активности и спецификацију имена табела и имена
поља у бази, како би независне целине, имплементиране од стране различитих делова
развојног тима, на крају рада биле успешно интегрисане.Развојном тиму документ служи 
као основа за дизајн и имплементацију.

\subsection{Организација документа}

Остатак документа организован је у следећа поглавља:
\begin{enumerate}
  \item Модел података – модел података у бази и шема базе
  \item Табеле – списак табела
\end{enumerate}

\subsection{Речник појмова и скраћеница}

\begin{itemize}
  \item IE – \textit{Information Engineering}, нотација за моделовање података
  \item ER – \textit{Entity/Relationship}, нотација за моделовање података
\end{itemize}

\subsection{Отворена питања}

\noindent
\setcellgapes{4pt}
\makegapedcells
\begin{tabularx}{\linewidth}{|l|X|X|}
    \hline
    \textbf{Редни број} & \textbf{Опис} & \textbf{Решење} \\
    \hline
    1 & Да ли треба раздвојити табелу CollectedMovie на две табеле за прегледане филмове и филмове из листе жеља или их оставити у једној табели? & \\
    \hline
    & & \\
    \hline
\end{tabularx}
