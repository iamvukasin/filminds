\section{Опис производа}

\subsection{Преглед архитектуре система}

Сервис је замишљен као веб-апликација која се покреће на серверу који подржава \textit{Python} и
реализована је уз помоћ библиотеке \textit{Django}. Комуникација између корисничког интерфејса и
позадинског сервера одрађена је коришћењем технологије \textit{Ajax}.

Подаци о корисницима, филмовима, предлозима експерата и прегледима филмова се чувају у
релационој бази података. За потребе овог система користи се \textit{PostgreSQL} база података.

Основни подаци о филмовима се дохватају из екстерне базе филмова \textit{The Movie Database} преко
понуђеног API-ја. За препознавање корисничке откуцане текстуалне поруке, бот користи \textit{Wit.ai} API,
који уз помоћ технологије за обраду природног језика враћа структурисани одговор о значењу анализиране
текстуалне поруке.

\subsection{Преглед карактеристика}

\noindent
\setcellgapes{4pt}
\makegapedcells
\begin{tabularx}{\linewidth}{|X|X|}
    \hline
    \textbf{Корист за корисника} & \textbf{Карактеристика која је обезбеђује} \\
    \hline
    Брз и једноставан приступ захтеваним информацијама & Једноставни кориснички интерфејс и
    централизовани приступ подацима преко комуникације са ботом \\
    \hline
    Персонализовани предлози филмова & Праћење историје прегледа и чувања филмова на сервису
    од стране корисника и коришћење екстерних база филмова за боље рангирање филмова \\
    \hline
    Сигурност података и сервиса & Ауторизација корисника и разликовање различитих врста
    корисника са различитим степеном привилегија \\
    \hline
    Мултиплатформски приступ сервису & Коришћење модерних веб-технологија (HTML5, CSS3) које
    омогућавају корисницима униформни приступ са разних врста уређаја \\
    \hline
\end{tabularx}
