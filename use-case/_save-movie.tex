\section{Увод}

\subsection{Резиме}

Документ дефинише сценарио употребе функционалности чувања филмова од стране тренутно пријављеног
корисника.

\subsection{Намена документа и циљне групе}

Документ ће користити чланови тима који буду радили на имплементацији и тестирању овог случаја
употребе. Такође, документ се може користити и при писању корисничког упутства за коришћење
сервиса.

\subsection{Референце}

\begin{enumerate}
    \item Пројектни задатак
    \item Упутство за писање спецификације сценарија употребе
    \item Guidelines – Use Case, Rational Unified Process 2000
    \item Guidelines – Use Case Storyboard, Rational Unified Process 2000
\end{enumerate}

\subsection{Отворена питања}

\noindent
\setcellgapes{4pt}
\makegapedcells
\begin{tabularx}{\linewidth}{|l|X|X|}
    \hline
    \textbf{Редни број} & \textbf{Опис} & \textbf{Решење} \\
    \hline
    & & \\
    \hline
\end{tabularx}

\section{Сценарио употребе функционалности чувања филмова}

\subsection{Кратак опис}

Како би се обезбедио персонализовани приступ, регистрованим корисницима омогућено је чување одгледаних
филмова и филмова које желе да одгледају. Пријављени корисник у било ком тренутку може затражити додавање филма у
неку од ове две листе филмова. Ова опција је омогућена свим регистрованим корисницима, укључујући и експерте 
и администраторе. Потребно је пратити овакве корисничке акције и приликом сваког додавања филма у овe листe бележити те информације у бази података.

\subsection{Ток догађаја}

\subsubsection{Корисник бира опцију да сачува филм у листу одгледаних филмова током дописивања са ботом}

\begin{enumerate}
    \item Корисник у менију са стране бира опцију \textit{Chat}.
    \item Корисник тражи предлог филмова слањем поруке.
    \item Систем приказује предлоге филмова на основу корисникових захтева. Сваки филм приказује се у картици 
    у оквиру које постоји дугме за додавање филма у листу већ одгледаних.
    \item Корисник кликом на предвиђено дугме додаје филм у листу одгледаних филмова.
\end{enumerate}

\subsubsection{Корисник бира опцију да сачува филм у листу жеља током дописивања са ботом}
Аналогни кораци као у 2.2.1. У кораку 3. постоји аналогно дугме ѕа додавање у листу жеља.

\subsubsection{Корисник бира опцију да сачува филм у листу одгледаних филмова током прегледања предлога експерата}

\begin{enumerate}
    \item Корисник у менију са стране бира опцију \textit{Expert picks}.
    \item Систем приказује форму за приказ листе филмова препоручених од стране експерата.
    \item Корисник бира категорију експерта чије предлоге жели да погледа избором опције из падајућег менија.
    \item Систем приказује листу филмова препоручених од стране изабране категорије експерта. Сваки филм приказује 
    се као ред у табели у оквиру ког постоји дугме за додавање филма у листу одгледаних филмова.
    \item Корисник кликом на предвиђено дугме додаје филм у листу одгледаних филмова.
\end{enumerate}

\subsubsection{Корисник бира опцију да сачува филм у листу жеља током прегледања предлога експерата}

Аналогни кораци као у 2.2.3. У кораку 4. постоји аналогно дугме ѕа додавање у листу жеља.

\subsubsection{Корисник бира опцију да премести филм раније обележен за гледање у листу одгледаних}

\begin{enumerate}
    \item Корисник у менију са стране бира опцију \textit{Favourites}.
    \item Систем приказује филмове сачуваних у листу жеља. Сваки филм приказује се у картици 
    у оквиру које постоји дугме за додавање филма у листу одгледаних.
    \item Корисник кликом на предвиђено дугме премешта филм у листу одгледаних филмова.
\end{enumerate}

\subsection{Посебни захтеви}

Нема посебних захтева.

\subsection{Предуслови}

\begin{itemize}
    \item Корисник мора бити регистрован на сервису и у том тренутку пријављен. 
    \item У случају 2.2.5 корисник мора имати бар један филм сачуван у листи жеља како би могао 
    да га премести у листу одгледаних.
\end{itemize}

\subsection{Последице}

При сваком додавању филма у неку од две листе ради се упис у базу. Додатно, при премештању филма раније 
обележеног за гледање у листу одгледаних ради се и брисање у оквиру базе (тај филм више није у листи жеља за гледање).
Такође, ово може утицати на статистику филмова.

