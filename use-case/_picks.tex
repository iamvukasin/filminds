\section{Увод}

\subsection{Резиме}

Документ дефинише сценарио употребе функционалности прегледа предлога филомова, предложених од стране експерата.

\subsection{Намена документа и циљне групе}

Документ ће користити чланови тима који буду радили на имплементацији и тестирању овог случаја
употребе. Такође, документ се може користити и при писању корисничког упутства за коришћење
сервиса.

\subsection{Референце}

\begin{enumerate}
    \item Пројектни задатак
    \item Упутство за писање спецификације сценарија употребе
    \item Guidelines – Use Case, Rational Unified Process 2000
    \item Guidelines – Use Case Storyboard, Rational Unified Process 2000
\end{enumerate}

\subsection{Отворена питања}

\noindent
\setcellgapes{4pt}
\makegapedcells
\begin{tabularx}{\linewidth}{|l|X|X|}
    \hline
    \textbf{Редни број} & \textbf{Опис} & \textbf{Решење} \\
    \hline
    & & \\
    \hline
\end{tabularx}

\section{Сценарио употребе функционалности прегледа филмова предложених од стране експерата}

\subsection{Кратак опис}

Како би се обезбедила додатна погодност регистрованим корисницима, омогућено им је прегледање листе филмова предложених од стране експерата. Екперти за филмове су корисници унапред предодређени од стране администратора. Сваки експерт
има категорију филмова за коју је задужен. Пријављени корисник у било ком тренутку може затражити преглед предлога експерта било које доступне категорије. Ова опција је омогућена свим регистрованим корисницима, укључујући и експерте и администраторе. Информације се приказују на захтев корисника.

\subsection{Ток догађаја}

\subsubsection{Главни сценарио}

\begin{enumerate}
    \item Корисник у менију са стране бира опцију \textit{Expert picks}.
    \item Систем приказује форму за приказ листе филмова препоручених од стране експерата.
    \item Корисник бира категорију експерта чије предлоге жели да погледа избором опције из падајућег менија.
    \item Систем приказује листу филмова препоручених од стране изабране категорије експерта. Сваки филм приказује 
    се као ред у табели у оквиру ког постоји дугме за додавање филма у листу жеља филмова и листу одгледаних филмова 
    или преглед више информација о филму.
\end{enumerate}

\subsection{Посебни захтеви}

Нема посебних захтева.

\subsection{Предуслови}

\begin{itemize}
    \item Корисник мора бити регистрован на сервису и у том тренутку пријављен.
\end{itemize}

\subsection{Последице}

Нема никаквих последица.

