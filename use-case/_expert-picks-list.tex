\documentclass[12pt,a4paper]{article}
\usepackage[margin=1in]{geometry}
\usepackage[T1]{fontenc}
\usepackage[serbianc]{babel}
\usepackage{fullwidth}
\usepackage{tabularx}
\usepackage{makecell}
\usepackage{enumitem}

% section prefixes
\makeatletter
\renewcommand{\@seccntformat}[1]{%
  \ifcsname prefix@#1\endcsname
    \csname prefix@#1\endcsname
  \else
    \csname the#1\endcsname\quad
  \fi}
% define \prefix@section
\newcommand\prefix@section{\thesection. }
\makeatother

\begin{document}

\begin{titlepage}
\begin{center}
  Универзитет у Београду \\
  Електротехнички факултет \\
  Катедра за рачунарску технику и информатику \\
  \vfill

  {\fontsize{50}{60}\selectfont Filminds}
  \vskip 0.6cm

  {\large Спецификација сценарија употребе функционалности ажурирања листе предлога експерата }
  \vskip 0.3cm
  
  {\large Тим: Super Trio Mario}
  \vskip 0.3cm

  {\large Верзија 1.0}

  \vfill
  \vfill

  Март 2019.
\hfill

\end{center}
\end{titlepage}

\section*{Историја измена}
\noindent
\setcellgapes{4pt}
\makegapedcells
\begin{tabularx}{\linewidth}{|l|l|X|X|}
    \hline
    \textbf{Датум} & \textbf{Верзија} & \textbf{Кратак опис} & \textbf{Аутор} \\
    \hline
    23.5.2019. & 1.0 & Иницијална верзија & Немања Дивнић \\
    \hline
    & & & \\
    \hline
\end{tabularx}
\newpage

\tableofcontents
\newpage

\section{Увод}
\subsection{Резиме}
Документ дефинише сценарио употребе ажурирања листе предлога експерата.
\subsection{Намена документа и циљне групе}
Документ ће користити чланови тима који буду радили на имплементацији и тестирању овог случаја употребе. Такође, документ се може користити и при писању корисничког упутства за коришћење сервиса.
\subsection{Референце}
 
\begin{enumerate}
  \item Пројектни задатак
  \item Упутство за писање спецификације сценарија употребе
  \item Guidelines - Use Case, Rational Unified Process 2000
  \item Guidelines - Use Case Storyboard, Rational Unified Process 2000
\end{enumerate}

\subsection{Отворена питања}

\noindent
\setcellgapes{4pt}
\makegapedcells
\begin{tabularx}{\linewidth}{|l|X|X|}
    \hline
    \textbf{Редни број} & \textbf{Опис} & \textbf{Решење} \\
    \hline
     &  &  \\
    \hline
    & &  \\
    \hline
\end{tabularx}
\section{Сценарио употребе за ажурирање листе предлога експерата}
\subsection{Кратак опис}
Сваки експерт има категорију за коју је задужен. У сваком тренутку он може да дода филм у листу, обрише филм из листе и да мења ранг филма на листи. Када заврши са радом, сачува све измене кликом на дугме предвиђеног за то.



\subsection{Ток догађаја}

   \subsubsection{Главни ток догађаја} 
 \begin{enumerate}
        \item {Корисник се налази на прозору \textit{Add expert picks}.}
        \item {Корисник ажурира листу.}
        \item {Корисник притисне дугме (стрелицу) за чување измена.}
        \item {Систем чува нову листу.} \newline
   
\end{enumerate}

\subsubsection{Алтернативни ток: Корисник жели да дода филм} 
 \begin{enumerate}
        \item[2а.1.] Корисник притиска дугме \textit{Add movie}
        \item[2а.2.] Корисник уноси назив филма.   
        \item[2а.3.] Корисник притиска дугме \textit{Add}.
        \item[2а.4.] Систем приказује додати филм на прозору.
        \item[2а.5.] Повратак на корак 2.
\end{enumerate}

\subsubsection{Алтернативни ток: Корисник одустаје од додавања новог филма} 
 \begin{enumerate}
        \item[2б.1.] Корисник притиска дугме \textit{Add movie}
        \item[2б.2.] Корисник уноси назив филма.   
        \item[2б.3.] Корисник притиска дугме \textit{Cancel}.
        \item[2б.4.] Повратак на корак 2.
\end{enumerate}

\subsubsection{Алтернативни ток: Корисник жели да дода непостојећи филм} 
 \begin{enumerate}
        \item[2в.1.] Корисник притиска дугме \textit{Add movie}
        \item[2в.2.] Корисник уноси назив филма.   
        \item[2в.3.] Корисник притиска дугме \textit{Add}.
        \item[2в.4.] Систем приказује поруку о грешци.
        \item[2в.5.] Повратак на корак 2.
\end{enumerate}

\subsubsection{Алтернативни ток: Корисник мења ранг филма} 
 \begin{enumerate}
        \item[2г.1.] Корисник притиска стрелицу на горе или на доле
        \item[2г.2.] Систем приказује нови ранг филма.
        \item[2г.3.] Повратак на корак 2.
\end{enumerate}

\subsubsection{Алтернативни ток: Корисник брише филм} 
 \begin{enumerate}
        \item[2д.1.] Корисник притиска стрелицу на горе или на доле
        \item[2д.2.] Систем уклања филм из листе.
        \item[2д.3.] Повратак на корак 2.
\end{enumerate}

\subsubsection{Алтернативни ток: Корисник није направио промене} 
 \begin{enumerate}
        \item[4а.1.]Корисник добија поруку да није правио промене.
        \item[4а.2.] Повратак на корак 1.
\end{enumerate}

\subsection{Посебни захтеви}
Нема посебних захтева.
\subsection{Предуслови}
Корисник је улогован на систем, има титулу експерта и категорију за коју је експерт
 
 
\subsection{Последице}
Ажурирана листа се чува у базу. 

\end{document}
