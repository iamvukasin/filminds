\documentclass[12pt,a4paper]{article}
\usepackage[margin=1in]{geometry}
\usepackage[T1]{fontenc}
\usepackage[serbianc]{babel}
\usepackage{fullwidth}
\usepackage{tabularx}
\usepackage{makecell}
\usepackage{enumitem}

% section prefixes
\makeatletter
\renewcommand{\@seccntformat}[1]{%
  \ifcsname prefix@#1\endcsname
    \csname prefix@#1\endcsname
  \else
    \csname the#1\endcsname\quad
  \fi}
% define \prefix@section
\newcommand\prefix@section{\thesection. }
\makeatother

\begin{document}

\begin{titlepage}
\begin{center}
  Универзитет у Београду \\
  Електротехнички факултет \\
  Катедра за рачунарску технику и информатику \\
  \vfill

  {\fontsize{50}{60}\selectfont Filminds}
  \vskip 0.6cm

  {\large Спецификација сценарија употребе функционалности регистрације корисника  }
  \vskip 0.3cm
  
  {\large Тим: Super Trio Mario}
  \vskip 0.3cm

  {\large Верзија 1.0}

  \vfill
  \vfill

  Март 2019.
\hfill

\end{center}
\end{titlepage}

\section*{Историја измена}
\noindent
\setcellgapes{4pt}
\makegapedcells
\begin{tabularx}{\linewidth}{|l|l|X|X|}
    \hline
    \textbf{Датум} & \textbf{Верзија} & \textbf{Кратак опис} & \textbf{Аутор} \\
    \hline
    13.3.2019. & 1.0 & Иницијална верзија & Немања Дивнић \\
    \hline
    & & & \\
    \hline
\end{tabularx}
\newpage

\tableofcontents
\newpage

\section{Увод}
\subsection{Резиме}
Документ дефинише сценарио употребе за регистрацију корисника.
\subsection{Намена документа и циљне групе}
Документ ће користити чланови тима који буду радили на имплементацији и тестирању овог случаја употребе. Такође, документ се може користити и при писању корисничког упутства за коришћење сервиса.
\subsection{Референце}
 
\begin{enumerate}
  \item Пројектни задатак
  \item Упутство за писање спецификације сценарија употребе
  \item Guidelines - Use Case, Rational Unified Process 2000
  \item Guidelines - Use Case Storyboard, Rational Unified Process 2000
\end{enumerate}

\subsection{Отворена питања}

\noindent
\setcellgapes{4pt}
\makegapedcells
\begin{tabularx}{\linewidth}{|l|X|X|}
    \hline
    \textbf{Редни број} & \textbf{Опис} & \textbf{Решење} \\
    \hline
    1. & Да ли треба одмах пријавити корисника, уколико је регистрација успешна? &  \\
    \hline
    & &  \\
    \hline
\end{tabularx}
\section{Сценарио регистрације корисника}
\subsection{Кратак опис}
Уколико корисник не поседује налог, а жели да користи напредније могућности сервиса, потребно је да
се региструје уношењем личних података, као и бирањем одговарајуће шифре, ради боље безбедности налога. Унети
подаци се чувају у бази података и користе се за ауторизацију корисника приликом пријављивања.
\newpage
\subsection{Ток догађаја}

   \subsubsection{Главни ток догађаја} 
 \begin{enumerate}
        \item {Корисник се налази на прозору за регистрацију.}\newline
         Пред корисником су пет поља, вертикално распоређена (одозго на доле:  \textit{Name},  \textit{Username},  \textit{Email}, \textit{Password} и  \textit{Repeat password}). Испод њих је дугме \textit{Register}. 
        \item {Корисник уноси тражене податке.}
        \item {Корисник притиска дугме \textit{Register}.}
        \item {Регистрација је успешна.}\newline
            Уколико је корисник унео све податке, што подразумева име и презиме, корисничко име и мејл, који нису већ заузети, а у два поља за шифру унео идентичан скуп карактера, добиће поруку да је успешно регистрован.
\end{enumerate}
\subsubsection {Алтернативни ток: Корисничко име или мејл су већ заузети}
 \begin{enumerate}
    \item[4а.1.] Приликом регистрације, врше се провере на јединственост мејла и корисничког имена. У овом случају, неки од ова два податка већ постоје у бази, стога ће кориснику бити онемогућена регистрација, а он ће добити информацију, који од ова два податка је заузет.
    \item[4а.2.] Систем исписује поруку о грешци. 
    \item[4а.3.] Повратак на корак 2.
    \end{enumerate}
   \subsubsection {Алтернативни ток: Унесени подаци су невалидни}
     \begin{enumerate}
    \item[4б.1.]Уколико корисник не унесе све податке који се од њега траже, мејл је некоректног облика, корисничко име садржи недозвољене карактере или је у пољима за шифру унео различите скупове карактера, корисник ће бити обавештен да је унео невалидне податке и добиће нову шансу да се региструје коректно.
    \item[4б.2.] Систем исписује поруку о грешци. 
    \item[4б.3.] Повратак на корак 2.
    \end{enumerate}
\subsection{Посебни захтеви}
Нема посебних захтева.
\subsection{Предуслови}
 \begin{itemize}
     \item Корисник не сме бити пријављен на систем у том тренутку.
 \end{itemize}
 
 
\subsection{Последице}
У случају успешне регистрације, сви подаци које је корисник дао, биће сачувани у базу, а он ће моћи да их искористи за пријаву на сервис и коришћење напреднијих могућности.

\end{document}
