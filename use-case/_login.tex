\documentclass[12pt,a4paper]{article}
\usepackage[margin=1in]{geometry}
\usepackage[T1]{fontenc}
\usepackage[serbianc]{babel}
\usepackage{fullwidth}
\usepackage{tabularx}
\usepackage{makecell}
\usepackage{enumitem}

% section prefixes
\makeatletter
\renewcommand{\@seccntformat}[1]{%
  \ifcsname prefix@#1\endcsname
    \csname prefix@#1\endcsname
  \else
    \csname the#1\endcsname\quad
  \fi}
% define \prefix@section
\newcommand\prefix@section{\thesection. }
\makeatother

\begin{document}

\begin{titlepage}
\begin{center}
  Универзитет у Београду \\
  Електротехнички факултет \\
  Катедра за рачунарску технику и информатику \\
  \vfill

  {\fontsize{50}{60}\selectfont Filminds}
  \vskip 0.6cm

  {\large Спецификација сценарија употребе функционалности ауторизације и пријављивања корисника }
  \vskip 0.3cm
  
  {\large Тим: Super Trio Mario}
  \vskip 0.3cm

  {\large Верзија 1.0}

  \vfill
  \vfill

  Март 2019.
\hfill

\end{center}
\end{titlepage}

\section*{Историја измена}
\noindent
\setcellgapes{4pt}
\makegapedcells
\begin{tabularx}{\linewidth}{|l|l|X|X|}
    \hline
    \textbf{Датум} & \textbf{Верзија} & \textbf{Кратак опис} & \textbf{Аутор} \\
    \hline
    13.3.2019. & 1.0 & Иницијална верзија & Немања Дивнић \\
    \hline
    & & & \\
    \hline
\end{tabularx}
\newpage

\tableofcontents
\newpage

\section{Увод}
\subsection{Резиме}
Документ дефинише сценарио употребе за ауторизацију и пријављивања корисника.
\subsection{Намена документа и циљне групе}
Документ ће користити чланови тима који буду радили на имплементацији и тестирању овог случаја употребе. Такође, документ се може користити и при писању корисничког упутства за коришћење сервиса.
\subsection{Референце}
 
\begin{enumerate}
  \item Пројектни задатак
  \item Упутство за писање спецификације сценарија употребе
  \item Guidelines - Use Case, Rational Unified Process 2000
  \item Guidelines - Use Case Storyboard, Rational Unified Process 2000
\end{enumerate}

\subsection{Отворена питања}

\noindent
\setcellgapes{4pt}
\makegapedcells
\begin{tabularx}{\linewidth}{|l|X|X|}
    \hline
    \textbf{Редни број} & \textbf{Опис} & \textbf{Решење} \\
    \hline
     &  &  \\
    \hline
\end{tabularx}
\section{Сценарио ауторизације и пријављивања корисника}
\subsection{Кратак опис}Регистровани корисници се пријављују коришћењем свог корисничког имена и шифре. Унети подаци морају да се
подударају са вредностима који су сачувани у бази података.

\subsubsection{Администратор}
 Администратор може да користи бота за захтеве предлагања филмова, да
прати историју прегледаних и сачуваних филмова, да прегледа издвојене филмове од стране експерата, као и да
поставља и уклања експерте и прати статистику о коришћењу сајта и прегледаним и сачуваним филмовима.
\subsubsection{Експерт}
Експерт може да користи бота за захтеве предлагања филмова, да
прати историју прегледаних и сачуваних филмова, да прегледа издвојене филмове од стране експерата, као и да
додаје и уклања филмове са експертске листе филмова своје категорије и прати статистику о прегледаним и
сачуваним филмовима.

\subsubsection{Регистровани корисник}
Регистровани корисник, који није експерт и није администратор,
може да користи бота за захтеве предлагања филмова, да прати историју прегледаних и сачуваних филмова и
да прегледа издвојене филмове од стране експерата.
\subsection{Ток догађаја}

   \subsubsection{Главни ток догађаја} 
 \begin{enumerate}
        \item {Корисник се налази на прозору за пријављивање.}\newline
          Пред гостом су два поља, вертикално распоређена (одозго на доле: \textit{Username} и \textit{Password}), док је испод тих поља дугме \textit{Login}. При дну екрана налази се линк \textit{Proceed without login}.
      
        \item {Корисник уноси тражене податке.}
        \item {Корисник притиска дугме \textit{Login}.}
        \item {Пријављивање је успешно.}\newline
            Када корисник унесе валидне податке, што подразумева корисничко име и одговарајућу лозинку, биће успешно пријављен.
   
\end{enumerate}
\subsubsection {Алтернативни ток: Пријављивање је неуспешно }
 \begin{enumerate}
    \item[4а.1.]  Уколико корисник не унесе корисничко име или лозинку, унесе корисничко име које не постоји у бази или унесе лозинку, која није одговарајућа за корисничко име, добиће поруку да је унео невалидне податке.
    \item[4а.2.] Систем исписује поруку о грешци.
    \item[4а.3.] Повратак на корак 2.
    \end{enumerate}
\subsection{Посебни захтеви}
Нема посебних захтева.
\subsection{Предуслови}
 \begin{itemize}
     \item Корисник не сме бити пријављен на систем у том тренутку.
 \end{itemize}
 
 
\subsection{Последице}
Нема последица.

\end{document}
