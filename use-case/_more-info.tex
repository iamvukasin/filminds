\section{Увод}

\subsection{Резиме}

Документ дефинише сценарио употребе функционалности прегледа више информација о филму.

\subsection{Намена документа и циљне групе}

Документ ће користити чланови тима који буду радили на имплементацији и тестирању овог случаја
употребе. Такође, документ се може користити и при писању корисничког упутства за коришћење
сервиса.

\subsection{Референце}

\begin{enumerate}
    \item Пројектни задатак
    \item Упутство за писање спецификације сценарија употребе
    \item Guidelines – Use Case, Rational Unified Process 2000
    \item Guidelines – Use Case Storyboard, Rational Unified Process 2000
\end{enumerate}

\subsection{Отворена питања}

\noindent
\setcellgapes{4pt}
\makegapedcells
\begin{tabularx}{\linewidth}{|l|X|X|}
    \hline
    \textbf{Редни број} & \textbf{Опис} & \textbf{Решење} \\
    \hline
    & & \\
    \hline
\end{tabularx}

\section{Сценарио употребе функционалности прегледа више информација о филму}

\subsection{Кратак опис}

Корисник приликом било ког приказа филма може да затражи приказ више информација о истом. Он се приказује само на кориснички захтев.

\subsection{Ток догађаја}

\subsubsection{Корисник бира опцију за приказ више информација о филму током дописивања са ботом}

\begin{enumerate}
    \item Корисник у менију са стране бира опцију \textit{Chat}.
    \item Корисник тражи предлог филмова слањем поруке.
    \item Систем приказује предлоге филмова на основу корисникових захтева. Сваки филм приказује се у картици 
    у оквиру које постоји дугме за приказ више информација.
    \item Корисник кликом на предвиђено дугме отвара страницу са прегледом информација о филму.
\end{enumerate}

\subsubsection{Корисник бира опцију за приказ више информација о филму током прегледања листе жеља}

\begin{enumerate}
    \item Корисник у менију са стране бира опцију \textit{Favourites}.
    \item Систем приказује листу жеља филмова сачуваних од стране тренутног корисника. Сваки филм 
    приказује се у картици у оквиру које постоји дугме за приказ више информација.
    \item Корисник кликом на предвиђено дугме отвара страницу са прегледом информација о филму.
\end{enumerate}

\subsubsection{Корисник бира опцију за приказ више информација о филму током прегледања листе одгледаних филмова}

\begin{enumerate}
    \item Корисник у менију са стране бира опцију \textit{Watched}.
    \item Систем приказује листу одгледаних филмова сачуваних од стране тренутног корисника. Сваки филм 
    приказује се у картици у оквиру које постоји дугме за приказ више информација.
    \item Корисник кликом на предвиђено дугме отвара страницу са прегледом информација о филму.
\end{enumerate}

\subsubsection{Корисник бира опцију за приказ више информација о филму током прегледања предлога експерата}

\begin{enumerate}
    \item Корисник у менију са стране бира опцију \textit{Expert picks}.
    \item Систем приказује форму за приказ листе филмова препоручених од стране експерата.
    \item Корисник бира категорију експерта чије предлоге жели да погледа избором опције из падајућег менија.
    \item Систем приказује листу филмова препоручених од стране изабране категорије експерта. Сваки филм приказује 
    се као ред у табели у оквиру ког постоји дугме за преглед више информација о филму.
    \item Корисник кликом на предвиђено дугме отвара страницу са прегледом информација о филму.
\end{enumerate}

\subsection{Посебни захтеви}

Нема посебних захтева.

\subsection{Предуслови}

\begin{itemize}
    \item У случајевима 2.2.2, 2.2.3 и 2.2.4 корисник мора бити улогован у систему. 
    \item У случајевима 2.2.2 и 2.2.3 корисник мора имати бар неки сачуван филм у одговарајућој листи, 
    како би могао да затражи преглед више информација за њега.
\end{itemize}

\subsection{Последице}

Нема последица.

