\documentclass[12pt,a4paper]{article}
\usepackage[margin=1in]{geometry}
\usepackage[T1]{fontenc}
\usepackage[serbianc]{babel}
\usepackage{fullwidth}
\usepackage{tabularx}
\usepackage{makecell}
\usepackage{enumitem}

% section prefixes
\makeatletter
\renewcommand{\@seccntformat}[1]{%
  \ifcsname prefix@#1\endcsname
    \csname prefix@#1\endcsname
  \else
    \csname the#1\endcsname\quad
  \fi}
% define \prefix@section
\newcommand\prefix@section{\thesection. }
\makeatother

\begin{document}

\begin{titlepage}
\begin{center}
  Универзитет у Београду \\
  Електротехнички факултет \\
  Катедра за рачунарску технику и информатику \\
  \vfill

  {\fontsize{50}{60}\selectfont Filminds}
  \vskip 0.6cm

  {\large Спецификација сценарија употребе функционалности ауторизације и пријављивања гостију }
  \vskip 0.3cm
  
  {\large Тим: Super Trio Mario}
  \vskip 0.3cm

  {\large Верзија 1.0}

  \vfill
  \vfill

  Март 2019.
\hfill

\end{center}
\end{titlepage}

\section*{Историја измена}
\noindent
\setcellgapes{4pt}
\makegapedcells
\begin{tabularx}{\linewidth}{|l|l|X|X|}
    \hline
    \textbf{Датум} & \textbf{Верзија} & \textbf{Кратак опис} & \textbf{Аутор} \\
    \hline
    12.3.2019. & 1.0 & Иницијална верзија & Немања Дивнић \\
    \hline
    & & & \\
    \hline
\end{tabularx}
\newpage

\tableofcontents
\newpage

\section{Увод}
\subsection{Резиме}
Документ дефинише сценарио употребе за ауторизацију и пријављивања гостију.
\subsection{Намена документа и циљне групе}
Документ ће користити чланови тима који буду радили на имплементацији и тестирању овог случаја употребе. Такође, документ се може користити и при писању корисничког упутства за коришћење сервиса.
\subsection{Референце}
 
\begin{enumerate}
  \item Пројектни задатак
  \item Упутство за писање спецификације сценарија употребе
  \item Guidelines - Use Case, Rational Unified Process 2000
  \item Guidelines - Use Case Storyboard, Rational Unified Process 2000
\end{enumerate}

\subsection{Отворена питања}

\noindent
\setcellgapes{4pt}
\makegapedcells
\begin{tabularx}{\linewidth}{|l|X|X|}
    \hline
    \textbf{Редни број} & \textbf{Опис} & \textbf{Решење} \\
    \hline
    & &  \\
    \hline
    & &  \\
    \hline
\end{tabularx}
\section{Сценарио ауторизације и пријављивања гостију }
\subsection{Кратак опис}

Гости (нерегистровани корисници) се не пријављују коришћењем свог корисничког имена и шифре, већ им је приступ
сервису омогућен, али лимитиран. Гост може само да користи бота за захтеве предлагања филмова.

\subsection{Ток догађаја}

   \subsubsection{Главни ток догађаја} 
 \begin{enumerate}
        \item {Корисник се налази на прозору за пријављивање.}\newline
          Пред гостом су два поља, вертикално распоређена (одозго на доле:  \textit{Username} и \textit{Password}), док је испод тих поља дугме \textit{Login}. При дну екрана налази се линк  \textit{Proceed without login}.
        \item {Корисник кликне на линк \textit{Proceed without login}.}
        \item {Појављује се екран за дописивање.}
   
\end{enumerate}

\subsection{Посебни захтеви}
Нема посебних захтева.
\subsection{Предуслови}
 \begin{itemize}
     \item Корисник не сме бити пријављен на систем у том тренутку.
 \end{itemize}
 
\subsection{Последице}
Нема последица.

\end{document}
