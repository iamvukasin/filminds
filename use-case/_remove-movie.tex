\section{Увод}

\subsection{Резиме}

Документ дефинише сценарио употребе функционалности брисања филмова из листе жеља за гледање и листе 
одгледаних филмова од стране тренутно пријављеног корисника.

\subsection{Намена документа и циљне групе}

Документ ће користити чланови тима који буду радили на имплементацији и тестирању овог случаја
употребе. Такође, документ се може користити и при писању корисничког упутства за коришћење
сервиса.

\subsection{Референце}

\begin{enumerate}
    \item Пројектни задатак
    \item Упутство за писање спецификације сценарија употребе
    \item Guidelines – Use Case, Rational Unified Process 2000
    \item Guidelines – Use Case Storyboard, Rational Unified Process 2000
\end{enumerate}

\subsection{Отворена питања}

\noindent
\setcellgapes{4pt}
\makegapedcells
\begin{tabularx}{\linewidth}{|l|X|X|}
    \hline
    \textbf{Редни број} & \textbf{Опис} & \textbf{Решење} \\
    \hline
    & & \\
    \hline
\end{tabularx}

\section{Сценарио употребе функционалности брисања филмова из листе сачуваних}

\subsection{Кратак опис}

Како би се обезбедио персонализовани приступ, регистрованим корисницима омогућено је чување одгледаних
филмова и филмова које желе да одгледају. Пријављени корисник у било ком тренутку може затражити уклањање неког 
филма из неке од ове две листе филмова. Ова опција је омогућена свим регистрованим корисницима, укључујући и експерте 
и администраторе. Филм се брише из неке од листа на захтев корисника.

\subsection{Ток догађаја}

\subsubsection{Корисник бира опцију да уклони филм из листе одгледаних филмова}

\begin{enumerate}
    \item Корисник у менију са стране бира опцију \textit{Watched}.
    \item Систем приказује листу одгледаних филмова сачуваних од стране тренутног корисника. Сваки филм 
    приказује се у картици у оквиру које постоји дугме за брисање филма из листе одгледаних филмова.
    \item Корисник кликом на предвиђено дугме уклања филм из листе одгледаних филмова.
\end{enumerate}

\subsubsection{Корисник бира опцију да уклони филм из листе жеља филмова}

\begin{enumerate}
    \item Корисник у менију са стране бира опцију \textit{Favourites}.
    \item Систем приказује листу жеља филмова сачуваних од стране тренутног корисника. Сваки филм 
    приказује се у картици у оквиру које постоји дугме за брисање филма из листе жеља за гледање.
    \item Корисник кликом на предвиђено дугме уклања филм из листе жеља филмова.
\end{enumerate}

\subsection{Посебни захтеви}

Нема посебних захтева.

\subsection{Предуслови}

\begin{itemize}
    \item Корисник мора бити регистрован на сервису и у том тренутку пријављен. 
    \item Kорисник има бар један филм сачуван у одговарајућој листи како би могао да га обрише.
\end{itemize}

\subsection{Последице}

При уклањању филма из неке од ове две листе ради се брисање у оквиру базе (тај филм више није у тој листи). Такође,
ово може утицати на статистику филмова.

