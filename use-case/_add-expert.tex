\section{Увод}

\subsection{Резиме}

Документ дефинише сценарио употребе за додавање експерта који предлаже филмове за листу
филмова одређене категорије.

\subsection{Намена документа и циљне групе}

Документ ће користити чланови тима који буду радили на имплементацији и тестирању овог случаја
употребе. Такође, документ се може користити и при писању корисничког упутства за коришћење
сервиса.

\subsection{Референце}

\begin{enumerate}
    \item Пројектни задатак
    \item Упутство за писање спецификације сценарија употребе
    \item Guidelines – Use Case, Rational Unified Process 2000
    \item Guidelines – Use Case Storyboard, Rational Unified Process 2000
\end{enumerate}

\subsection{Отворена питања}

\noindent
\setcellgapes{4pt}
\makegapedcells
\begin{tabularx}{\linewidth}{|l|X|X|}
    \hline
    \textbf{Редни број} & \textbf{Опис} & \textbf{Решење} \\
    \hline
    & & \\
    \hline
\end{tabularx}

\section{Сценарио додавања експерта}

\subsection{Кратак опис}

Експерти су задужени за уређивање једне листе о предлогу филмова за одређену категорију. Како
би се лакше управљало експертима, администраторима је додељена улога о додељивању експерата за
одређену категорију, а самим тим и потенцијално промовисање стандарних регистрованих корисника
у експерте. Овај сценарио употребе дефинише начин додавања експерта унутар администраторског
управљачког простора.

\subsection{Ток догађаја}

\subsubsection{Главни сценарио}

\begin{enumerate}
    \item Корисник бира опцију \textit{Admin dashboard} (администраторски управљачки простор) у менију.
    \item Систем приказује табелу корисника који су постављени за експерте у одређеним категоријама.
    \item Корисник притиска дугме \textit{Add expert}.
    \item Систем приказује прозор-дијалог за унос имејла или корисничког имена другог корисника
      који треба да се промовише у улогу експерта и категорију листе предлога која још није
      додељена ниједном експерту.
    \item Корисник уписује тражене податке и шаље захтев за додавање експерта.
    \item Систем одобрава захтев.
\end{enumerate}

\subsubsection{Алтернативни ток: Корисник одустаје од захтева}

\begin{enumerate}
    \item [5a.1.] Корисник одустаје од уноса притиском да дугме \textit{Cancel} и захтев се не шаље.
    \item [5а.2.] Повратак на корак 2.
\end{enumerate}

\subsubsection{Алтернативни ток: Корисник уноси непостојећег корисника}

\begin{enumerate}
    \item [6а.1.] Систем одбија захтев, јер не постоји корисник са унетим имејлом или корисничким
      именом.
    \item [6а.2.] Систем исписује поруку о грешци.
    \item [6а.3.] Повратак на корак 5.
\end{enumerate}

\subsubsection{Алтернативни ток: Корисник уноси администратора}

\begin{enumerate}
    \item [6б.1.] Систем одбија захтев, јер унети корисник је администратор.
    \item [6б.2.] Систем исписује поруку о грешци.
    \item [6б.3.] Повратак на корак 5.
\end{enumerate}

\subsubsection{Алтернативни ток: Корисник уноси експерта који је већ има додељену категорију}

\begin{enumerate}
    \item [6в.1.] Систем одбија захтев, јер унети корисник је експерт који је већ има додељену
      категорију.
    \item [6в.2.] Систем исписује поруку о грешци.
    \item [6в.3.] Повратак на корак 5.
\end{enumerate}

\subsection{Посебни захтеви}

Нема посебних захтева.

\subsection{Предуслови}

\begin{itemize}
    \item Корисник мора бити регистрован на сервису и у том тренутку пријављен.
    \item Корисник мора имати привилегије администратора.
\end{itemize}

\subsection{Последице}

Додељивање експерта одређеној категорији препорука филмова се осликава и на стање система
сачуваног у бази података. У случају да је додати корисник пре додавања био само регистровани
корисник без било каквих привилегија, његова улога у систему се мења и он постаје експерт.
