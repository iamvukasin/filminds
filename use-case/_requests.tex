\documentclass[12pt,a4paper]{article}
\usepackage[margin=1in]{geometry}
\usepackage[T1]{fontenc}
\usepackage[serbianc]{babel}
\usepackage{fullwidth}
\usepackage{tabularx}
\usepackage{makecell}
\usepackage{enumitem}

% section prefixes
\makeatletter
\renewcommand{\@seccntformat}[1]{%
  \ifcsname prefix@#1\endcsname
    \csname prefix@#1\endcsname
  \else
    \csname the#1\endcsname\quad
  \fi}
% define \prefix@section
\newcommand\prefix@section{\thesection. }
\makeatother

\begin{document}

\begin{titlepage}
\begin{center}
  Универзитет у Београду \\
  Електротехнички факултет \\
  Катедра за рачунарску технику и информатику \\
  \vfill

  {\fontsize{50}{60}\selectfont Filminds}
  \vskip 0.6cm

  {\large Спецификација сценарија употребе функционалности захтева за препоруку филмова  }
  \vskip 0.3cm
  
  {\large Тим: Super Trio Mario}
  \vskip 0.3cm

  {\large Верзија 1.0}

  \vfill
  \vfill

  Март 2019.
\hfill

\end{center}
\end{titlepage}

\section*{Историја измена}
\noindent
\setcellgapes{4pt}
\makegapedcells
\begin{tabularx}{\linewidth}{|l|l|X|X|}
    \hline
    \textbf{Датум} & \textbf{Верзија} & \textbf{Кратак опис} & \textbf{Аутор} \\
    \hline
    13.3.2019. & 1.0 & Иницијална верзија & Немања Дивнић \\
    \hline
    & & & \\
    \hline
\end{tabularx}
\newpage

\tableofcontents
\newpage

\section{Увод}
\subsection{Резиме}
Документ дефинише сценарио употребе за захтеве корисника за препоруку филмова.
\subsection{Намена документа и циљне групе}
Документ ће користити чланови тима који буду радили на имплементацији и тестирању овог случаја употребе. Такође, документ се може користити и при писању корисничког упутства за коришћење сервиса.
\subsection{Референце}
 
\begin{enumerate}
  \item Пројектни задатак
  \item Упутство за писање спецификације сценарија употребе
  \item Guidelines - Use Case, Rational Unified Process 2000
  \item Guidelines - Use Case Storyboard, Rational Unified Process 2000
\end{enumerate}

\subsection{Отворена питања}

\noindent
\setcellgapes{4pt}
\makegapedcells
\begin{tabularx}{\linewidth}{|l|X|X|}
    \hline
    \textbf{Редни број} & \textbf{Опис} & \textbf{Решење} \\
    \hline
    1. & Да ли је потребно сугерисати кориснику, које питање да постави, како би се убрзало слање затхева? &  \\
    \hline
    & &  \\
    \hline
\end{tabularx}
\section{Сценарио употребе за захтеве корисника за препоруку филмова}
\subsection{Кратак опис}
Сви корисници могу да шаљу захтеве боту преко интерактивног екрана за дописивање. На захтев послатог у виду
поруке, који је на енглеском језику, бот шаље повратну поруку са списком предложених филмова.


\subsubsection{Захтев за препоруку популарних филмова}

Приликом захтевања листе предлога популарних филмова од стране корисника, бот враћа листу филмова који су
тренутно најпопуларнији по статистикама екстерних база филмова.

\subsubsection{Захтев за препоруку филмова одређеног жанра}

Приликом захтевања листе предлога филмова одређеног жанра од стране корисника, бот враћа листу филмова који
су захтеваног жанра и тренутно су најпопуларнији по статистикама екстерних база филмова.

\subsubsection{Захтев за препоруку филмова из одређене године}

Приликом захтевања листе предлога филмова из одређене године од стране корисника, бот враћа листу филмова који
су издати захтеване године и тренутно су најпопуларнији по статистикама екстерних база филмова.

\subsubsection{Комбиновани захтев за препоруку филмова}

Корисник може да захтева филмове комбинујући претходно поменуте захтеве у један захтев, при чему му бот враћа
листу филмова који испуњавају послати захев и тренутно су најпопуларнији по статистикама екстерних база филмова.


\subsection{Ток догађаја}

   \subsubsection{Главни ток догађаја} 
 \begin{enumerate}
        \item {Корисник се налази на прозору за \textit{Chat}.}
        \item {Корисник унесе поруку.}
        \item {Корисник притисне дугме (стрелицу) за слање поруке.}
        \item {Систем приказује тражену листу.} \newline
    Уколико је бот добро разумео захтев, односно препознао одговарајући тип захтева, одговориће на захтев листом филмова.
   
\end{enumerate}

\subsubsection{Алтернативни ток: Систем приказује погрешну листу} 
 \begin{enumerate}
        \item[4а.1.] 
    Бот ће одговорити на захтев листом филмова, међутим, та листа није у складу са очекиваним одговором.
   \item[4а.2.] Повратак на корак 2.
\end{enumerate}

\subsubsection{Алтернативни ток: бот не разуме захтев} 
 \begin{enumerate}
        \item[4б.]  
      Уколико бот није успео да из захтева извуче податке, који су му неопходни, на исти ће одговорити да није разумео захтев.
       \item[4б.2.] Повратак на корак 2.
\end{enumerate}
\subsection{Посебни захтеви}
Нема посебних захтева.
\subsection{Предуслови}
Нема предуслова.
 
 
\subsection{Последице}
Корисник ће моћи поново да пошаље нови захтев на исти начин као што је и раније слао. Филмови који су претраживани се уносе у базу, због утицаја на статистику.

\end{document}
