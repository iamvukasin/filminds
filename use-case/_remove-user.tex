\section{Увод}

\subsection{Резиме}

Документ дефинише сценарио употребе за уклањање регистрованог корисника који нема администраторске
привилегије.

\subsection{Намена документа и циљне групе}

Документ ће користити чланови тима који буду радили на имплементацији и тестирању овог случаја
употребе. Такође, документ се може користити и при писању корисничког упутства за коришћење
сервиса.

\subsection{Референце}

\begin{enumerate}
    \item Пројектни задатак
    \item Упутство за писање спецификације сценарија употребе
    \item Guidelines – Use Case, Rational Unified Process 2000
    \item Guidelines – Use Case Storyboard, Rational Unified Process 2000
\end{enumerate}

\subsection{Отворена питања}

\noindent
\setcellgapes{4pt}
\makegapedcells
\begin{tabularx}{\linewidth}{|l|X|X|}
    \hline
    \textbf{Редни број} & \textbf{Опис} & \textbf{Решење} \\
    \hline
    1. & Да ли дефинитивно уклањати корисника или га само маркирати као обрисаног? & \\
    \hline
    & & \\
    \hline
\end{tabularx}

\section{Сценарио брисања корисника}

\subsection{Кратак опис}

Администратору, као одговорном особљу за добар рад сервиса, додељена је улога уклањања било ког
корисника који није администратор. Администратор ову опцију треба да користи само у посебним
ситуацијама прописаним у правилима и начину коришћења сервиса.

\subsection{Ток догађаја}

\subsubsection{Главни сценарио}

\begin{enumerate}
    \item Корисник бира опцију \textit{Admin dashboard} (администраторски управљачки простор) у менију.
    \item Корисник уноси имејл или корисничко име корисника који треба да се избрише.
    \item Корисник притиска дугме за брисање чиме потврђује одлуку.
    \item Систем брише корисника.
\end{enumerate}

\subsubsection{Алтернативни ток: Корисник уноси непостојећег корисника}

\begin{enumerate}
    \item [4а.1.] Систем одбија захтев, јер не постоји корисник са унетим имејлом или корисничким
      именом.
    \item [4а.2.] Систем исписује поруку о грешци.
    \item [4а.3.] Повратак на корак 2.
\end{enumerate}

\subsubsection{Алтернативни ток: Корисник уноси администратора}

\begin{enumerate}
    \item [4б.1.] Систем одбија захтев, јер унети корисник је администратор.
    \item [4б.2.] Систем исписује поруку о грешци.
    \item [4б.3.] Повратак на корак 2.
\end{enumerate}

\subsection{Посебни захтеви}

Нема посебних захтева.

\subsection{Предуслови}

\begin{itemize}
    \item Корисник мора бити регистрован на сервису и у том тренутку пријављен.
    \item Корисник мора имати привилегије администратора.
    \item Мора постојати бар један корисник који може да се уклони.
\end{itemize}

\subsection{Последице}

Корисник се уклања из система, што се осликава и на стање базе. Такође, сви сачувани и прегледани
филмови тог корисника се бришу. Ово има утицај и на свеукупну статистику сајта, као и на
статистику o филмовима.
